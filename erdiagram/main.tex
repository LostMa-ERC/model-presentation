\documentclass[a3paper, landscape]{report}
\usepackage{layout}
\usepackage[T1]{fontenc}
\usepackage{er-diagram}
\usepackage{tabularx,booktabs}
\usepackage{caption}
\usepackage{hyperref}
\usepackage{xcolor}


\begin{document}

    % \begin{tikzpicture}
    %     \node (tex) [] {\Large\textbf{Story}};
    %     \node (obj) [right=of tex, xshift=0.7\textwidth] {\Large{\textbf{Object}}};
    %     \draw (tex) edge[<->] (obj);
    %     \node (ai) [above=of tex.center, yshift=-1em] {\Large\textit{Immaterial}};
    %     \node (a2) [above=of obj.center, yshift=-1em] {\Large\textit{Material}};
    % \end{tikzpicture}

    % \vspace*{1em}

    \begin{center}
        \tikzstyle{s} = [rectangle, rounded corners, minimum width=2cm, text width=3cm, minimum height=1cm, text centered, draw=black]
\tikzstyle{arrow} = [thick,->,>=stealth]
\begin{tikzpicture}[-,shorten >=1pt,auto,node distance=1.5cm,semithick]
\tikzstyle{every state}=[fill=red,draw=none,text=white]

\pic [] { entity = {{cycle}
{\textbf{Cycle}}
{
  \colorbox{gray}{\textsc{Label Components}} \\
  - \textbf{main title} \\
  \hline \colorbox{gray}{\textsc{Description}} \\
  - alternative titles \textsuperscript{+} \\
  - is part of (\texttt{Cycle}) \\
  - matter [\texttt{Matter}] \\
  \hline \colorbox{gray}{\textsc{References}} \\
  - is referenced in \textsuperscript{+} (\texttt{Citation}) \\
}
}};

\pic [right = 3.3em of cycle] { entity = {{work}
{\textbf{Work}}
{
  \colorbox{gray}{\textsc{Label Components}} \\
  - \textbf{main title} \\
  - is part of (\texttt{Cycle})\\
  \hline \colorbox{gray}{\textsc{Description}} \\
  - alternative titles \textsuperscript{+} \\
  - is modeled on \textsuperscript{+} (\texttt{Work})\\
  \hline \colorbox{gray}{\textsc{References}} \\
  - is referenced in \textsuperscript{+} (\texttt{Citation}) \\
}
}};

\pic [right = 3.3em of work] { entity = {{text}
{\textbf{Text}}
{
  \colorbox{gray}{\textsc{Label Components}} \\
  - \textbf{main title} \\
  - language \\
  - form [\texttt{Form}] \\
  \hline \colorbox{gray}{\textsc{Hypothesis}} \\
  - hypothetical (y/n) \\
  - hypothesis \textsuperscript{+} (\texttt{Hypothesis}) \\
  \hline \colorbox{gray}{\textsc{Description}} \\
  - \textbf{is expression of} (\texttt{Work})\\
  - alternative titles \textsuperscript{+} \\
  - is modeled on \textsuperscript{+} (\texttt{Text})\\
  - \textbf{matter} \textsuperscript{+} [\texttt{Matter}] \\
  - \textbf{regional genre}\\
  - poetic meter\\
  - rhyme type \\
  - strophe count \\
  - strophe length \\
  - verse length \\
  \hline \colorbox{gray}{\textsc{Production}} \\
  - is written by (\texttt{Person})\\
  - is adapted by (\texttt{Person})\\
  - \textbf{date of creation} [\texttt{Date}] \\
  - date of creation source \textsuperscript{+} (\texttt{Citation})\\
  - place of creation (\texttt{Place}) \\
  - place of creation source \textsuperscript{+} (\texttt{Citation}) \\
}
}};

\pic [right = 3.3em of text] { entity = {{witness}
{\textbf{Witness}}
{
  \colorbox{gray}{\textsc{Label Components}} \\
  - \textbf{is manifestation of} (\texttt{Text}) \\
  - main siglum \\
  - has part \textsuperscript{+} (\texttt{TextPart}) \\
  \hline \colorbox{gray}{\textsc{Hypothesis}} \\
  - hypothetical (y/n) \\
  - hypothesis \textsuperscript{+} (\texttt{Hypothesis}) \\
  - was part of (\texttt{Document}) \\
  \hline \colorbox{gray}{\textsc{Description}} \\
  - status [\texttt{Status}] \\
  - is citation (y/n) \\
  - is manifestation of (\texttt{Text})\\
  - is preceded by fragm. (\texttt{Witness})\\
  - is preceded by wit. (\texttt{Witness})\\
  - is referenced in (\texttt{Citation}) \\
  \hline
  \colorbox{gray}{\textsc{Production}} \\
  - scribe \textsuperscript{+} (\texttt{Person})\\
  - written language variant, aka scripta\\
  - \textbf{date of creation} [\texttt{Date}] \\
  - date of creation source \textsuperscript{+} (\texttt{Citation})\\
  - place of creation (\texttt{Place}) \\
  - place of creation source \textsuperscript{+} (\texttt{Citation}) \\
}
}};

\pic [right = 3.3em of witness] { entity = {{part}
{\textbf{Part}}
{
  \colorbox{gray}{\textsc{Label Components}} \\
  - \textbf{is inscribed on} (\texttt{Document}) \\
  - order in series of parts (default 1) \\
  \hline \colorbox{gray}{\textsc{Locus}} \\
  - number of lines represented from \texttt{Text} \\
  - part of \texttt{Text} represented \\
  - volume number in multi-volume \texttt{Witness} \\
  \hline \colorbox{gray}{\textsc{Extent}} \\
  - number of lines \\
  - lines are incomplete, i.e. cut (y/n) \\
  - first page | last page \textsuperscript{+} (fol\textsubscript{1}|fol\textsubscript{2})
}
}};

\pic [right = 3.3em of part] { entity = {{doc}
{\textbf{Document}}
{
  \colorbox{gray}{\textsc{Label Components}} \\
  - current shelfmark \\
  - is conserved in (\texttt{Repository}) \\
  - invented label if hypothetical \\
  \hline \colorbox{gray}{\textsc{Hypothesis}} \\
  - hypothetical (y/n) \\
  - hypothesis \textsuperscript{+} (\texttt{Hypothesis}) \\
  \hline \colorbox{gray}{\textsc{Description}} \\
  - collection\\
  - old shelfmark \textsuperscript{+}\\
  - has digitization (\texttt{Digitization})\\
  - collection of fragments (y/n) \\
  - is referenced in \textsuperscript{+} (\texttt{Citation})\\
}
}};

\pic [below = 3em of part] {entity = {{images}
{\textbf{Images}}
{
  \colorbox{gray}{\textsc{Label Components}}\\
  - \textbf{presents} (\texttt{Part})\\
  - \textbf{contained in} (\texttt{Digitization})\\
  \colorbox{gray}{\textsc{Description}} \\
  - \textbf{first view | last view} (view\textsubscript{1}|view\textsubscript{2}) \\
  - \textbf{corresponding first page | last page} (fol\textsubscript{1}|fol\textsubscript{2}) \\
}
}};

\pic [right = 3.3em of images] { entity = {{digitization}
{\textbf{Digitization}}
{
  \colorbox{gray}{\textsc{Label Components}} \\
  - \textbf{URL} \\
  - ARK \\
  \hline \colorbox{gray}{\textsc{Description}} \\
  - digitized by (\texttt{Repository})
  - link to IIIF manifest \\
}
}};

\pic [right = 3.3em of doc] { entity = {{repository}
{\textbf{Repository}}
{
  \colorbox{gray}{\textsc{Label Components}} \\
  - \textbf{institutional name} \\
  - \textbf{City} (\texttt{Place}) \\
  \hline \colorbox{gray}{\textsc{Description}} \\
  - alternative names \textsuperscript{+}\\
  - VIAF identifier \\
  - ISNI \\
  - Biblissima identifier \\
  - website \\
}
}};

\pic at (current bounding box.north east) [anchor=north east, yshift=6em] { entityassociative = {{place}
{\textbf{Place}}
{
  \colorbox{gray}{\textsc{Label Components}} \\
  - \textbf{Place name} \\
  - State or Province or Department \\
  - \textbf{Country} \\
  \hline \colorbox{gray}{\textsc{Description}} \\
  - Place type \\
  - Location mappable \\
  - Location certainty (1, 2, 3, 4, 5) \\
  - GeoNames ID \\
}
}};

\pic at (current bounding box.north west) [anchor=north west, yshift=2em] { entityassociative = {{hypothesis}
{\textbf{Hypothesis}}
{
  \colorbox{gray}{\textsc{Label Components}} \\
  - \textbf{attested object} (\texttt{Text}, \texttt{Wintess}, \texttt{Document}) \\
  - \textbf{related real objects} \textsuperscript{+} (\texttt{Work}, \texttt{Text}, \texttt{Witness}, \texttt{Part}) \\
  \hline \colorbox{gray}{\textsc{Description}} \\
  - claim \\
  - certainty (1, 2, 3, 4, 5) \\
  - is referenced in \textsuperscript{+} (\texttt{Citation}) \\
  - additional note \\
}
}};

\pic  at (current bounding box.south east) [anchor=south east] { entityassociative = {{person}
{{\textbf{Person}}}
{
  \colorbox{gray}{\textsc{Label Components}} \\
  - name \\
  \hline \colorbox{gray}{\textsc{Description}} \\
  - alternative names \textsuperscript{+} \\
  - fictional (y/n) \\
  - is referenced in \textsuperscript{+} (\texttt{Citation})\\
}
}};

\pic at (current bounding box.south west) [anchor=south west]{ entityassociative = {{citation}
{\textbf{Citation}}
{
  \colorbox{gray}{\textsc{Label Components}} \\
  - author \\
  - title or source \\
  - year [\texttt{Date}] \\
  \hline \colorbox{gray}{\textsc{Description}} \\
  - DOI or URL \\
  - BibTeX citation \\
}
}};


\pic [right = 6em of citation, yshift=6em] { entityweak = {{matter}
{\textbf{Matter}}
{
 - Britain \\
 - France \\
 - Rome \\
 - other \\
}
}};

\pic [below = 1em of matter] { entityweak = {{form}
{\textbf{Form}}
{
 - prose \\
 - verse \\
 - mixed \\
}
}};

\pic [below = 1em of form] { entityweak = {{status}
{\textbf{Status}}
{
 - complete \\
 - defective \\
 - fragment \\
}
}};


\draw[oone-many] (cycle) -- (work);
\draw[one-many] (work) -- (text);
\draw[one-omany] (text) -- (witness);
\draw[one-omany] (witness) -- (part);
\draw[many-one] (part) -- (doc);
\draw[omany-one] (doc) -- (repository);
\draw[omany-omany] (witness.80) |-++(5,1)-| node[pos=0.4]{\textit{Witness(es) of Hypothetical Document}} (doc);


\draw[omany-oone] (digitization.35) -|++(1,1)-| (repository.south);
\draw[omany-one] (digitization.north) -- (doc.south);
\draw[omany-one] (images) -- (part);
\draw[many-one] (images.east) -- (digitization.west);

% Place associative entity
\draw[omany-oone] (text.80) |-++(1,3.5)-| node[]{} (place.80);
\draw[omany-oone] (witness.90) |-++(1,2.5)-| node[]{} (place.100);
\draw[one-omany] (place) -- (repository);

% Person associative entity
\draw[omany-oone] (text.south) |-++(1,-5)-| node[]{} (person.290);
\draw[omany-oone] (witness.south) |-++(1,-4)-| node[]{} (person.250);

% Hypothesis associative entity
\draw[omany-oone] (witness.100) |-++(0,3.5)-| node[]{} (hypothesis.110);
\draw[omany-oone] (work.90) |-++(0,1)|- node[]{} (hypothesis.330);
\draw[omany-oone] (text.100) |-++(0,0.5)|- node[]{} (hypothesis.30);
\draw[omany-oone] (part.north) |-(2,11)-| node[]{} (hypothesis.70);


\end{tikzpicture}
    \end{center}

    \begin{table}[b]
        \caption*{Legend}
        \begin{tabularx}{\textwidth}{|l|l|l|l|X|}
        \hline
        \footnotesize{\textbf{Bold} = required}
        & \footnotesize{\textsuperscript{+} = repeatable}
        & \footnotesize{(\texttt{Entity}) = foreign key / the value is the linked entity}
        & \footnotesize{[\texttt{Table}]} = the value is one of the listed terms
        & \footnotesize{[\texttt{Date}] = a Heurist date object (i.e. the year 1270, the day 15/03/1270, a range 1240-1270, etc.) with fields for a comment and for a degree of certainty}\\
        \hline
    \end{tabularx}
    \end{table}

\end{document}
